% Generated by Sphinx.
\def\sphinxdocclass{report}
\documentclass[letterpaper,10pt,english]{sphinxmanual}
\usepackage[utf8]{inputenc}
\DeclareUnicodeCharacter{00A0}{\nobreakspace}
\usepackage[T1]{fontenc}
\usepackage{babel}
\usepackage{times}
\usepackage[Bjarne]{fncychap}
\usepackage{longtable}
\usepackage{sphinx}
\usepackage{multirow}


\title{SADIT Documentation}
\date{September 23, 2012}
\release{1.0.0}
\author{Jing Conan Wang}
\newcommand{\sphinxlogo}{}
\renewcommand{\releasename}{Release}
\makeindex

\makeatletter
\def\PYG@reset{\let\PYG@it=\relax \let\PYG@bf=\relax%
    \let\PYG@ul=\relax \let\PYG@tc=\relax%
    \let\PYG@bc=\relax \let\PYG@ff=\relax}
\def\PYG@tok#1{\csname PYG@tok@#1\endcsname}
\def\PYG@toks#1+{\ifx\relax#1\empty\else%
    \PYG@tok{#1}\expandafter\PYG@toks\fi}
\def\PYG@do#1{\PYG@bc{\PYG@tc{\PYG@ul{%
    \PYG@it{\PYG@bf{\PYG@ff{#1}}}}}}}
\def\PYG#1#2{\PYG@reset\PYG@toks#1+\relax+\PYG@do{#2}}

\def\PYG@tok@gd{\def\PYG@tc##1{\textcolor[rgb]{0.63,0.00,0.00}{##1}}}
\def\PYG@tok@gu{\let\PYG@bf=\textbf\def\PYG@tc##1{\textcolor[rgb]{0.50,0.00,0.50}{##1}}}
\def\PYG@tok@gt{\def\PYG@tc##1{\textcolor[rgb]{0.00,0.25,0.82}{##1}}}
\def\PYG@tok@gs{\let\PYG@bf=\textbf}
\def\PYG@tok@gr{\def\PYG@tc##1{\textcolor[rgb]{1.00,0.00,0.00}{##1}}}
\def\PYG@tok@cm{\let\PYG@it=\textit\def\PYG@tc##1{\textcolor[rgb]{0.25,0.50,0.56}{##1}}}
\def\PYG@tok@vg{\def\PYG@tc##1{\textcolor[rgb]{0.73,0.38,0.84}{##1}}}
\def\PYG@tok@m{\def\PYG@tc##1{\textcolor[rgb]{0.13,0.50,0.31}{##1}}}
\def\PYG@tok@mh{\def\PYG@tc##1{\textcolor[rgb]{0.13,0.50,0.31}{##1}}}
\def\PYG@tok@cs{\def\PYG@tc##1{\textcolor[rgb]{0.25,0.50,0.56}{##1}}\def\PYG@bc##1{\colorbox[rgb]{1.00,0.94,0.94}{##1}}}
\def\PYG@tok@ge{\let\PYG@it=\textit}
\def\PYG@tok@vc{\def\PYG@tc##1{\textcolor[rgb]{0.73,0.38,0.84}{##1}}}
\def\PYG@tok@il{\def\PYG@tc##1{\textcolor[rgb]{0.13,0.50,0.31}{##1}}}
\def\PYG@tok@go{\def\PYG@tc##1{\textcolor[rgb]{0.19,0.19,0.19}{##1}}}
\def\PYG@tok@cp{\def\PYG@tc##1{\textcolor[rgb]{0.00,0.44,0.13}{##1}}}
\def\PYG@tok@gi{\def\PYG@tc##1{\textcolor[rgb]{0.00,0.63,0.00}{##1}}}
\def\PYG@tok@gh{\let\PYG@bf=\textbf\def\PYG@tc##1{\textcolor[rgb]{0.00,0.00,0.50}{##1}}}
\def\PYG@tok@ni{\let\PYG@bf=\textbf\def\PYG@tc##1{\textcolor[rgb]{0.84,0.33,0.22}{##1}}}
\def\PYG@tok@nl{\let\PYG@bf=\textbf\def\PYG@tc##1{\textcolor[rgb]{0.00,0.13,0.44}{##1}}}
\def\PYG@tok@nn{\let\PYG@bf=\textbf\def\PYG@tc##1{\textcolor[rgb]{0.05,0.52,0.71}{##1}}}
\def\PYG@tok@no{\def\PYG@tc##1{\textcolor[rgb]{0.38,0.68,0.84}{##1}}}
\def\PYG@tok@na{\def\PYG@tc##1{\textcolor[rgb]{0.25,0.44,0.63}{##1}}}
\def\PYG@tok@nb{\def\PYG@tc##1{\textcolor[rgb]{0.00,0.44,0.13}{##1}}}
\def\PYG@tok@nc{\let\PYG@bf=\textbf\def\PYG@tc##1{\textcolor[rgb]{0.05,0.52,0.71}{##1}}}
\def\PYG@tok@nd{\let\PYG@bf=\textbf\def\PYG@tc##1{\textcolor[rgb]{0.33,0.33,0.33}{##1}}}
\def\PYG@tok@ne{\def\PYG@tc##1{\textcolor[rgb]{0.00,0.44,0.13}{##1}}}
\def\PYG@tok@nf{\def\PYG@tc##1{\textcolor[rgb]{0.02,0.16,0.49}{##1}}}
\def\PYG@tok@si{\let\PYG@it=\textit\def\PYG@tc##1{\textcolor[rgb]{0.44,0.63,0.82}{##1}}}
\def\PYG@tok@s2{\def\PYG@tc##1{\textcolor[rgb]{0.25,0.44,0.63}{##1}}}
\def\PYG@tok@vi{\def\PYG@tc##1{\textcolor[rgb]{0.73,0.38,0.84}{##1}}}
\def\PYG@tok@nt{\let\PYG@bf=\textbf\def\PYG@tc##1{\textcolor[rgb]{0.02,0.16,0.45}{##1}}}
\def\PYG@tok@nv{\def\PYG@tc##1{\textcolor[rgb]{0.73,0.38,0.84}{##1}}}
\def\PYG@tok@s1{\def\PYG@tc##1{\textcolor[rgb]{0.25,0.44,0.63}{##1}}}
\def\PYG@tok@gp{\let\PYG@bf=\textbf\def\PYG@tc##1{\textcolor[rgb]{0.78,0.36,0.04}{##1}}}
\def\PYG@tok@sh{\def\PYG@tc##1{\textcolor[rgb]{0.25,0.44,0.63}{##1}}}
\def\PYG@tok@ow{\let\PYG@bf=\textbf\def\PYG@tc##1{\textcolor[rgb]{0.00,0.44,0.13}{##1}}}
\def\PYG@tok@sx{\def\PYG@tc##1{\textcolor[rgb]{0.78,0.36,0.04}{##1}}}
\def\PYG@tok@bp{\def\PYG@tc##1{\textcolor[rgb]{0.00,0.44,0.13}{##1}}}
\def\PYG@tok@c1{\let\PYG@it=\textit\def\PYG@tc##1{\textcolor[rgb]{0.25,0.50,0.56}{##1}}}
\def\PYG@tok@kc{\let\PYG@bf=\textbf\def\PYG@tc##1{\textcolor[rgb]{0.00,0.44,0.13}{##1}}}
\def\PYG@tok@c{\let\PYG@it=\textit\def\PYG@tc##1{\textcolor[rgb]{0.25,0.50,0.56}{##1}}}
\def\PYG@tok@mf{\def\PYG@tc##1{\textcolor[rgb]{0.13,0.50,0.31}{##1}}}
\def\PYG@tok@err{\def\PYG@bc##1{\fcolorbox[rgb]{1.00,0.00,0.00}{1,1,1}{##1}}}
\def\PYG@tok@kd{\let\PYG@bf=\textbf\def\PYG@tc##1{\textcolor[rgb]{0.00,0.44,0.13}{##1}}}
\def\PYG@tok@ss{\def\PYG@tc##1{\textcolor[rgb]{0.32,0.47,0.09}{##1}}}
\def\PYG@tok@sr{\def\PYG@tc##1{\textcolor[rgb]{0.14,0.33,0.53}{##1}}}
\def\PYG@tok@mo{\def\PYG@tc##1{\textcolor[rgb]{0.13,0.50,0.31}{##1}}}
\def\PYG@tok@mi{\def\PYG@tc##1{\textcolor[rgb]{0.13,0.50,0.31}{##1}}}
\def\PYG@tok@kn{\let\PYG@bf=\textbf\def\PYG@tc##1{\textcolor[rgb]{0.00,0.44,0.13}{##1}}}
\def\PYG@tok@o{\def\PYG@tc##1{\textcolor[rgb]{0.40,0.40,0.40}{##1}}}
\def\PYG@tok@kr{\let\PYG@bf=\textbf\def\PYG@tc##1{\textcolor[rgb]{0.00,0.44,0.13}{##1}}}
\def\PYG@tok@s{\def\PYG@tc##1{\textcolor[rgb]{0.25,0.44,0.63}{##1}}}
\def\PYG@tok@kp{\def\PYG@tc##1{\textcolor[rgb]{0.00,0.44,0.13}{##1}}}
\def\PYG@tok@w{\def\PYG@tc##1{\textcolor[rgb]{0.73,0.73,0.73}{##1}}}
\def\PYG@tok@kt{\def\PYG@tc##1{\textcolor[rgb]{0.56,0.13,0.00}{##1}}}
\def\PYG@tok@sc{\def\PYG@tc##1{\textcolor[rgb]{0.25,0.44,0.63}{##1}}}
\def\PYG@tok@sb{\def\PYG@tc##1{\textcolor[rgb]{0.25,0.44,0.63}{##1}}}
\def\PYG@tok@k{\let\PYG@bf=\textbf\def\PYG@tc##1{\textcolor[rgb]{0.00,0.44,0.13}{##1}}}
\def\PYG@tok@se{\let\PYG@bf=\textbf\def\PYG@tc##1{\textcolor[rgb]{0.25,0.44,0.63}{##1}}}
\def\PYG@tok@sd{\let\PYG@it=\textit\def\PYG@tc##1{\textcolor[rgb]{0.25,0.44,0.63}{##1}}}

\def\PYGZbs{\char`\\}
\def\PYGZus{\char`\_}
\def\PYGZob{\char`\{}
\def\PYGZcb{\char`\}}
\def\PYGZca{\char`\^}
\def\PYGZsh{\char`\#}
\def\PYGZpc{\char`\%}
\def\PYGZdl{\char`\$}
\def\PYGZti{\char`\~}
% for compatibility with earlier versions
\def\PYGZat{@}
\def\PYGZlb{[}
\def\PYGZrb{]}
\makeatother

\begin{document}

\maketitle
\tableofcontents
\phantomsection\label{index::doc}



\chapter{Introduction}
\label{index:introduction}\label{index:welcome-to-sadit-s-documentation}
SADIT is a byproduct of research project \textbf{A Coordinated Approach to
Cyber-Situation Awarness Based on Traffic Anomaly Detection}.  It is the
acronym of \textbf{S}ystematic \textbf{A}nomaly \textbf{D}etection of \textbf{I}nternet
\textbf{T}raffic.  The motivation of SADIT is to make the comparison and the
validation of internet anomaly deteciton algorithmes super easy.  It provides
a collection of Internet Anomaly Detection algorithms proposed by several
researchers. Now it contains:
\begin{enumerate}
\item {} 
Stochastic Anomaly Detector using Large Deviation Theory

\item {} 
Deterministic Temporal Anomaly Detector using Support Vector Machine

\item {} 
Deterministic Flow by Flow Detector using Support Vector Machine

\item {} 
Anomaly Detection Techniques using ART theory

\end{enumerate}

SADIT is not limited to the four algorithms listed above, its utimate goal is to
become a standard collection of a variety of internet anomaly detection
algorithms.

It also provides an easy-to-use tool to generate labeled flow records, which are
helpful in validatation and comparison of methods.

If you are a researcher interested in Internet Anomaly Detection, we strongly
encourage you to implement your algorithms following the APIs and data format of
SADIT so that you can easily compare your methods with exisiting algorithms in
SADIT. Your help will be highly appreciated if you can contribute your own
algorithm to the algorithm libray of SADIT. Feel free to contact me if you have
any question.


\chapter{What's New}
\label{index:what-s-new}
the version 1.0 is a result of big refactor of version 0.0. The refactor makes the code more
scalable and less buggy.
\begin{itemize}
\item {} 
\textbf{Paradigm of Object-oriented programming}: The \textbf{Configure} module and \textbf{Detector} module have been rewritten under
object-oriented paradigm. In version 0.0, all modules depends on the global
settings file setting.py, which make the code more vulunerable to bugs. In this verison only
few scripts depend on settings.py. Classes are widely used to reduce the need to
pass parameters around. In case that parameters passing is required, well-defined structures are used.

\item {} 
\textbf{Experiment}: A new folder ROOT/Experiment appears to contain different
experiments. You can write your own scripts of Experiment and put them in
this folder.

\item {} 
\textbf{Better Sensitivity Anaysis}: In the version 0.0, sensitivity anaysis is
done by change the global settings.py file and rerun the simulation. Since
settings.py is a typical python module,changing it during the run is really not
a good idea. In this version, special Experiment is designed to support
sensitivity analysis.

\end{itemize}


\chapter{Structure}
\label{index:structure}
\textbf{SADIT} consists of  two parts. The first part is a collection of anomalies
detection algorithms. The second part is labeled flow record generator. The
follow sections will describe the two parts accordingly.


\section{Collection of Anomaly Detection Algorithm}
\label{index:collection-of-anomaly-detection-algorithm}
All the detection algorithms locates in the \emph{ROOT/Detector} folder:
\begin{itemize}
\item {} 
\textbf{SVMDetector.py} contains two SVM based anomaly detection algorithmes
1. SVM Temporal Detector and 2. SVM Flow by Flow Detector.

\item {} 
\textbf{StoDetector.py} contains two anomaly detection algorithms based on
Large Deviation Theory.

\end{itemize}


\section{Labeled Flow Records Generator}
\label{index:labeled-flow-records-generator}
Labeled Flow Records Generator consists of a \emph{Configurer} and a \emph{Simulator}.
The \emph{Simulator} part is essentially a revised FS Simulator developed by
researchers at UW Madison. \emph{Configurer} first generate a flow specification (DOT
format) file with certain types of anomalies, then the \emph{Simulator} will generate
flow records and corresponding labels.


\subsection{Configurer}
\label{index:configurer}
\emph{Configurer} generate the corresponding DOT file according to description of user
behaviour. The important concepts in \emph{Configurer} are as follows:
\begin{itemize}
\item {} 
\textbf{Generator}: description of a certain type of flow traffic. For
examples, \emph{Harpoon} generator represents \href{http://cs.colgate.edu/~jsommers/harpoon/}{harpoon flows}.

\item {} 
\textbf{Behaviour}: description of temporal pattern. There are three types
behaviour:
\begin{itemize}
\item {} 
\textbf{Normal} behaviour is described by start time and duration.

\item {} \begin{description}
\item[{\textbf{I.I.D} behaviour has a list of possible states, but one state}] \leavevmode
will be selected as current state every \emph{t} seconds
according to certain probability distribution.

\end{description}

\item {} \begin{description}
\item[{\textbf{Markov} the state in different time is not independtly and}] \leavevmode
identically distributed,  but is a Markov process

\end{description}

\end{itemize}

\item {} 
\textbf{Modulator}: combine \emph{Behaviour} and \emph{Generator}, basicially description
of generator behaviour. There are three types of modulators, corresponding
to three behaviours described above.

\end{itemize}


\subsection{Simulator}
\label{index:simulator}
Simulator is basically a revised version of fs simulator. We have added
support to export anoumalous flows(add label information).


\chapter{Usage}
\label{index:usage}
To run SADIT, just go to the diretory of SADIT source code, change ROOT variable in
\textbf{settings.py} to the absolute path of the source directory. Then type

\begin{Verbatim}[commandchars=\\\{\}]
\$ ./run.py
\end{Verbatim}

in the command line. The help document will come out
\begin{description}
\item[{usage: run.py {[}-h{]} {[}-e EXPERIMENT{]} {[}-i INTERPRETER{]} {[}-d DETECT{]} {[}-m METHOD{]}}] \leavevmode
{[}--data\_handler DATA\_HANDLER{]} {[}--feature\_option FEATURE\_OPTION{]}
{[}--export\_flows EXPORT\_FLOWS{]}
{[}--entropy\_threshold ENTROPY\_THRESHOLD{]} {[}--pic\_name PIC\_NAME{]}

\end{description}

sadit
\begin{description}
\item[{optional arguments:}] \leavevmode\begin{optionlist}{3cm}
\item [-h, -{-}help]  
show this help message and exit
\item [-e EXPERIMENT, -{-}experiment EXPERIMENT]  
specify the experiment name you want to execute.
Experiments availiable are: {[}'Eval', `Experiment',
`test', `Sens', `DetectArgSearch', `MarkovSens',
`MultiSrvExperiment', `ImalseSettings',
`MarkovExperiment'{]}. An integrated experiment will run
fs-simulator first and use detector to detect the
result.
\item [-i INTERPRETER, -{-}interpreter INTERPRETER]  
--specify the interperter you want to use, now support
{[}cpython{]}, and {[}pypy{]}(only for detector)
\item [-d DETECT, -{-}detect DETECT]  
--detect {[}filename{]} will simply detect the flow file,
simulator will not run in this case, detector will
still use the configuration in the settings.py
\item [-m METHOD, -{-}method METHOD]  
--method {[}method{]} will specify the method to use.
Avaliable options are: {[}'svm\_fbf':
SVMFlowByFlowDetector SVM Flow By Flow Anomaly
Detector Method \textbar{} `mf': ModelFreeAnoDetector \textbar{} `mfmb':
FBAnoDetector model free and model based together \textbar{}
`svm\_temp': SVMTemporalDetector SVM Temporal
Difference Detector. Proposed by R.L Taylor.
Implemented by J. C. Wang \textless{}\href{mailto:wangjing@bu.ed}{wangjing@bu.ed}\textgreater{} \textbar{} `mb':
ModelBaseAnoDetector{]}
\item [-{-}data\_handler DATA\_HANDLER]  
--specify the data handler you want to use, the
availiable option are: {[}'fs\_deprec': DataFile from fs
output flow file to feature, this class is
\emph{depreciated} \textbar{} `fs': HardDiskFileHandler Data is
stored as Hard Disk File \textbar{} `xflow':
HardDiskFileHandler\_xflow \textbar{} `SperottoIPOM2009':
SQLDataFileHandler\_SperottoIPOM2009 ``Data File wrapper
for SperottoIPOM2009 format. it is store in mysql
server, visit \href{http://traces.simpleweb.org/traces/netfl}{http://traces.simpleweb.org/traces/netfl}
ow/netflow2/dataset\_description.txt for more
information \textbar{} `pcap2netflow':
HardDiskFileHandler\_pcap2netflow{]}
\item [-{-}feature\_option FEATURE\_OPTION]  
specify the feature option. feature option is a
dictionary describing the quantization level for each
feature. You need at least specify `cluster' and
`dist\_to\_center'. Note that, the value of `cluster' is
the cluster number. The avaliability of other features
depend on the data handler.
\item [-{-}export\_flows EXPORT\_FLOWS]  
specify the file name of exported abnormal flows.
Default is not export
\item [-{-}entropy\_threshold ENTROPY\_THRESHOLD]  
the threshold for entropy,
\item [-{-}pic\_name PIC\_NAME]  
picture name for the detection result
\end{optionlist}

\end{description}

The help document is quite clear, the only parameter I want clarify is
\emph{--experiment}. It specify the experiment you want to execute. An \textbf{experiment}
is actually a executable python script in Experiment folder.
..  I will take
..  \emph{Experiment.py} as an example to describle experiment.
Avaliable experiments as follows:
\begin{itemize}
\item {} 
\textbf{Experiment.py}: basic experiment which includes configuration,
simulation and detection, which corresponds to generating the
configuration files, generating the labeled flow records and detectng the
anomalies in the records, respectively.

\item {} 
\textbf{Sens.py}: Do sensistive analysis by change the degree of of anomalies,
run the detection algorithm accordingly and show results in the same
figure.

\item {} 
\textbf{MarkovExperiment.py}: Similar with Experiment, but for Markov type of
anomalies

\item {} 
\textbf{MarkovSens.py}: Sensitivity analysis of MarkovExperiment

\item {} 
\textbf{Eval.py}: Evaluation of the detection algorithmm calculate fpr, fnr and
plot the ROC curve

\item {} 
\textbf{DetectArgSearch.py}: runs detection algortihms with all combinations of
parameters and outputs the results to a folder, helps to select the
optimal parameters.

\end{itemize}

In addition to the parameters in the command line, SADIT has some more tunable
parameters in \textbf{ROOT/settings.py}. you can customize \textbf{SADIT} through changing
parameters in \textbf{settings.py} file. Since it is a typical python script, so you can
use any non-trival python sentence in the \textbf{settings.py}.


\section{Parameters for Labeled Flow Generator}
\label{index:parameters-for-labeled-flow-generator}
\begin{Verbatim}[commandchars=\\\{\}]
\PYG{n}{NET\PYGZus{}DESC} \PYG{o}{=} \PYG{n+nb}{dict}\PYG{p}{(}
        \PYG{n}{topo}\PYG{p}{,} \PYG{c}{\PYGZsh{} a list of list, the adjacent matrix of the topology}
        \PYG{n}{size}\PYG{p}{,} \PYG{c}{\PYGZsh{} no. of nodes in the network}
        \PYG{n}{srv\PYGZus{}list}\PYG{p}{,} \PYG{c}{\PYGZsh{} not used}
        \PYG{n}{link\PYGZus{}attr\PYGZus{}default}\PYG{p}{,} \PYG{c}{\PYGZsh{} default link attributed, delay, capacity, etc.}
        \PYG{n}{node\PYGZus{}type}\PYG{o}{=}\PYG{l+s}{'}\PYG{l+s}{NNode}\PYG{l+s}{'}\PYG{p}{,} \PYG{c}{\PYGZsh{} Node type, can be 'NNode', 'MarkovNode', etc.}
        \PYG{n}{node\PYGZus{}para}\PYG{p}{,} \PYG{c}{\PYGZsh{} not used.}
        \PYG{p}{)}
\end{Verbatim}

\begin{Verbatim}[commandchars=\\\{\}]
\PYG{n}{NORM\PYGZus{}DESC} \PYG{o}{=} \PYG{n+nb}{dict}\PYG{p}{(}
        \PYG{n}{TYPE}\PYG{p}{,} \PYG{c}{\PYGZsh{} can be 'NORMAL' or 'MARKOV'}
        \PYG{n}{start}\PYG{p}{,} \PYG{c}{\PYGZsh{} start time, should be string}
        \PYG{n}{node\PYGZus{}para}\PYG{p}{,} \PYG{c}{\PYGZsh{} node parameters for normal case}
        \PYG{n}{profile}\PYG{p}{,} \PYG{n}{DEFAULT\PYGZus{}PROFILE}\PYG{p}{,} \PYG{c}{\PYGZsh{} normal profile}
        \PYG{n}{src\PYGZus{}nodes}\PYG{p}{,} \PYG{c}{\PYGZsh{} node that will send traffic}
        \PYG{n}{dst\PYGZus{}nodes}\PYG{p}{,} \PYG{c}{\PYGZsh{} the destination of the traffic.}
        \PYG{p}{)}
\end{Verbatim}

\begin{Verbatim}[commandchars=\\\{\}]
\PYG{n}{ANO\PYGZus{}DESC} \PYG{o}{=} \PYG{n+nb}{dict}\PYG{p}{(}
        \PYG{n}{anoType}\PYG{p}{,} \PYG{c}{\PYGZsh{} anomaly type, can be anyone defined in  **ano\PYGZus{}map** of *Configure/API.py*,}
        \PYG{n}{ano\PYGZus{}node\PYGZus{}seq}\PYG{p}{,} \PYG{c}{\PYGZsh{} the sequence no. of the abnormal node}
        \PYG{l+s}{'}\PYG{l+s}{T}\PYG{l+s}{'}\PYG{p}{:}\PYG{p}{(}\PYG{l+m+mi}{2000}\PYG{p}{,} \PYG{l+m+mi}{3000}\PYG{p}{)}\PYG{p}{,} \PYG{c}{\PYGZsh{} time range of the anomaly}
        \PYG{o}{.}\PYG{o}{.}\PYG{p}{,} \PYG{c}{\PYGZsh{} anomaly specific  parameters}
        \PYG{p}{)}
\end{Verbatim}


\section{Parameters for Detectors}
\label{index:parameters-for-detectors}
\begin{Verbatim}[commandchars=\\\{\}]
\PYG{n}{DETECTOR\PYGZus{}DESC} \PYG{o}{=} \PYG{n+nb}{dict}\PYG{p}{(}
        \PYG{n}{interval}\PYG{o}{=}\PYG{l+m+mi}{20}\PYG{p}{,}
        \PYG{n}{win\PYGZus{}size}\PYG{o}{=}\PYG{l+m+mi}{200}\PYG{p}{,}
        \PYG{n}{win\PYGZus{}type}\PYG{o}{=}\PYG{l+s}{'}\PYG{l+s}{time}\PYG{l+s}{'}\PYG{p}{,} \PYG{c}{\PYGZsh{} 'time'\textbar{}'flow'}
        \PYG{n}{fr\PYGZus{}win\PYGZus{}size}\PYG{o}{=}\PYG{l+m+mi}{100}\PYG{p}{,} \PYG{c}{\PYGZsh{} window size for estimation of flow rate}
        \PYG{n}{false\PYGZus{}alarm\PYGZus{}rate} \PYG{o}{=} \PYG{l+m+mf}{0.001}\PYG{p}{,}
        \PYG{n}{unified\PYGZus{}nominal\PYGZus{}pdf} \PYG{o}{=} \PYG{n+nb+bp}{False}\PYG{p}{,} \PYG{c}{\PYGZsh{} used in sensitivities analysis}
        \PYG{n}{fea\PYGZus{}option} \PYG{o}{=} \PYG{p}{\PYGZob{}}\PYG{l+s}{'}\PYG{l+s}{dist\PYGZus{}to\PYGZus{}center}\PYG{l+s}{'}\PYG{p}{:}\PYG{l+m+mi}{2}\PYG{p}{,} \PYG{l+s}{'}\PYG{l+s}{flow\PYGZus{}size}\PYG{l+s}{'}\PYG{p}{:}\PYG{l+m+mi}{2}\PYG{p}{,} \PYG{l+s}{'}\PYG{l+s}{cluster}\PYG{l+s}{'}\PYG{p}{:}\PYG{l+m+mi}{3}\PYG{p}{\PYGZcb{}}\PYG{p}{,}
        \PYG{n}{ano\PYGZus{}ana\PYGZus{}data\PYGZus{}file} \PYG{o}{=} \PYG{n}{ANO\PYGZus{}ANA\PYGZus{}DATA\PYGZus{}FILE}\PYG{p}{,}
        \PYG{n}{normal\PYGZus{}rg} \PYG{o}{=} \PYG{n+nb+bp}{None}\PYG{p}{,} \PYG{c}{\PYGZsh{} range for normal traffic, if it none, use the}
        \PYG{n}{whole} \PYG{n}{traffic}
        \PYG{n}{detector\PYGZus{}type} \PYG{o}{=} \PYG{l+s}{'}\PYG{l+s}{mfmb}\PYG{l+s}{'}\PYG{p}{,} \PYG{c}{\PYGZsh{} can be anyone specfied by **detector\PYGZus{}map** in *Detector/API.py*}
        \PYG{n}{max\PYGZus{}detect\PYGZus{}num} \PYG{o}{=} \PYG{n+nb+bp}{None}\PYG{p}{,} \PYG{c}{\PYGZsh{} the maximum detection number the detector will run}
        \PYG{n}{data\PYGZus{}handler} \PYG{o}{=} \PYG{l+s}{'}\PYG{l+s}{fs}\PYG{l+s}{'}\PYG{p}{,} \PYG{c}{\PYGZsh{} specify the data handler, can be any value defined in}
        \PYG{n}{data\PYGZus{}handler\PYGZus{}map} \PYG{o+ow}{in} \PYG{n}{Detector}\PYG{o}{/}\PYG{n}{API}\PYG{o}{.}\PYG{n}{py}

        \PYG{c}{\PYGZsh{}\PYGZsh{}\PYGZsh{}\PYGZsh{} only for SVM approach \PYGZsh{}\PYGZsh{}\PYGZsh{}\PYGZsh{}\PYGZsh{}}
        \PYG{c}{\PYGZsh{} gamma = 0.01,}
        \PYG{p}{)}
\end{Verbatim}


\chapter{Want to implement your algorithm?}
\label{index:want-to-implement-your-algorithm}

\section{Use the labeled flow records generator in fs simulator}
\label{index:use-the-labeled-flow-records-generator-in-fs-simulator}
The generated flows will be the \emph{ROOT/Simulator} folder. The flows end with
\emph{\_flow.txt}, for example, n0\_flow.txt is the network flows trough node 0. File
start with \emph{abnormal\_} is the exported abnormal flows correspondingly.
\begin{description}
\item[{\textbf{A typical line is}}] \leavevmode
textexport n0 1348412129.925416 1348412129.925416 1348412130.070733 10.0.7.4:80-\textgreater{}10.0.8.5:53701 tcp 0x0 n1 5 4215 FSA

\item[{\textbf{line format}}] \leavevmode
prefix nodename time flow\_start\_time flow\_end\_time src\_ip:src\_port-\textgreater{}dst\_ip:dst\_port protocol payload destname unknown flowsize unknown

\end{description}

After finishing your detection algorihms, you need to add the corresponding
class name to \textbf{detector\_map} in \emph{ROOT/Detector/API.py}. Then you can implement
your own experiment to compare your algorithms with existing algorithms in
SADIT. Look at the sample examples in  \emph{ROOT/Experiemnt/} folder. Your can run
your experiment by typing

\begin{Verbatim}[commandchars=\\\{\}]
./run.py -e \textless{}Your Experiment Name\textgreater{}
\end{Verbatim}


\section{Use Other flow records}
\label{index:use-other-flow-records}
SADIT does not only support the text output format of fs simulator, but also
several other types of flow data. The handler classes locate in the
\emph{DataHandler.py} and \emph{DataHandler\_xflow.py}
\textbf{DataFile}
\begin{itemize}
\item {} 
\textbf{PreloadHardDistFile} hard disk flow file generated by fs-simulator. It
will preload all the flow file into memory, so it cannot deal with flow
file larger than your memery

\item {} 
\textbf{PreloadHardDistFile\_pcap2netflow} hard disk flow file generated by
\href{https://bitbucket.org/hbhzwj/pcap2netflow/src}{pcap2netflow} tool(the
format of \href{http://www.mindrot.org/projects/softflowd/}{flowd-reader})

\item {} 
\textbf{PreloadHardDistFile\_xflow}, hard disk flow file generated by xflow tool

\item {} 
\textbf{SQLFile\_SperottoIPOM2009}, labeled data stored in mysql server provided
by \href{http://traces.simpleweb.org/traces/netflow/netflow2/}{simpleweb.org}

\end{itemize}

The following class is the abstract base class for all data files

\begin{Verbatim}[commandchars=\\\{\}]
\PYG{k}{class} \PYG{n+nc}{Data}\PYG{p}{(}\PYG{n+nb}{object}\PYG{p}{)}\PYG{p}{:}
    \PYG{l+s+sd}{"""virtual base class for data. Data class deals with any implementation}
\PYG{l+s+sd}{    details of the data. it can be a file, a sql data base, and so on, as long}
\PYG{l+s+sd}{    as it support the pure virtual methods defined here.}
\PYG{l+s+sd}{    """}
    \PYG{k}{def} \PYG{n+nf}{get\PYGZus{}fea\PYGZus{}slice}\PYG{p}{(}\PYG{n+nb+bp}{self}\PYG{p}{,} \PYG{n}{rg}\PYG{o}{=}\PYG{n+nb+bp}{None}\PYG{p}{,} \PYG{n}{rg\PYGZus{}type}\PYG{o}{=}\PYG{n+nb+bp}{None}\PYG{p}{)}\PYG{p}{:}
        \PYG{l+s+sd}{""" get a slice of feature}
\PYG{l+s+sd}{        - **rg** is the range for the slice}
\PYG{l+s+sd}{        - **rg\PYGZus{}type** is the type for range, it can be ['flow'\textbar{}'time']}
\PYG{l+s+sd}{        """}
        \PYG{n}{abstract\PYGZus{}method}\PYG{p}{(}\PYG{p}{)}

    \PYG{k}{def} \PYG{n+nf}{get\PYGZus{}max}\PYG{p}{(}\PYG{n+nb+bp}{self}\PYG{p}{,} \PYG{n}{fea}\PYG{p}{,} \PYG{n}{rg}\PYG{o}{=}\PYG{n+nb+bp}{None}\PYG{p}{,} \PYG{n}{rg\PYGZus{}type}\PYG{o}{=}\PYG{n+nb+bp}{None}\PYG{p}{)}\PYG{p}{:}
        \PYG{l+s+sd}{""" get the max value of feature during a range}
\PYG{l+s+sd}{        - **fea** is a list of feature name}
\PYG{l+s+sd}{        - **rg** is the range}
\PYG{l+s+sd}{        - **rg\PYGZus{}type** is the range type}
\PYG{l+s+sd}{        the output is the a list of element which contains the max}
\PYG{l+s+sd}{        value of the feature in **fea**}
\PYG{l+s+sd}{        """}
        \PYG{n}{abstract\PYGZus{}method}\PYG{p}{(}\PYG{p}{)}
    \PYG{k}{def} \PYG{n+nf}{get\PYGZus{}min}\PYG{p}{(}\PYG{n+nb+bp}{self}\PYG{p}{,} \PYG{n}{fea}\PYG{p}{,} \PYG{n}{rg}\PYG{o}{=}\PYG{n+nb+bp}{None}\PYG{p}{,} \PYG{n}{rg\PYGZus{}time}\PYG{o}{=}\PYG{n+nb+bp}{None}\PYG{p}{)}\PYG{p}{:}
        \PYG{l+s+sd}{""" get min value of feature within a range. see **get\PYGZus{}max** for}
\PYG{l+s+sd}{        the meaning of the parameters}
\PYG{l+s+sd}{        """}
        \PYG{n}{abstract\PYGZus{}method}\PYG{p}{(}\PYG{p}{)}
\end{Verbatim}

It defines the operation we can do with data files. The four
algorithmes we implemented requires \textbf{get\_fea\_slice}, \textbf{get\_max} and
\textbf{get\_min} operations. You can implement more operations, but at least
implement these three operations.

Optionally, you can implement a handler class that will manipulate the
DataFile and and some useful quantities that may be useful to you algorithms.
For example, we implemented \textbf{HardDiskFileHandler} with get\_em() function to
get probability distribution of the flows, which is useful for the stochastic
approaches. If you just need the raw data, you can define simple handler class
with data file object you want to use as member variable.

Then you just need to add your data\_handler to \textbf{data\_handler\_handle\_map}
defined in \emph{ROOT/Detector/API.py}


\chapter{Download}
\label{index:download}
You can download sadit from \href{https://bitbucket.org/hbhzwj/sadit/get/2182e36f40d5.zip}{here}.

or you can user mercurial to get a complete copy with revision history

\begin{Verbatim}[commandchars=\\\{\}]
hg clone https://bitbucket.org/hbhzwj/sadit
\end{Verbatim}


\chapter{Installation}
\label{index:installation}
\textbf{SADIT} has been tested on python2.7.2.
this software depends on all softwares that fs-simulate depends on:
\begin{itemize}
\item {} 
ipaddr (2.1.1)  \href{http://ipaddr-py.googlecode.com/files/ipaddr-2.1.1.tar.gz}{Get}

\item {} 
networkx (1.0) \href{http://networkx.lanl.gov/download/networkx/networkx-1.0.1.tar.gz}{Get}

\item {} 
pydot (1.0.2) \href{http://pydot.googlecode.com/files/pydot-1.0.2.tar.gz}{Get}

\item {} 
pyparsing (1.5.2) \href{http://downloads.sourceforge.net/project/pyparsing/pyparsing/pyparsing-1.5.2/pyparsing-1.5.2.tar.gz?r=http\%3A\%2F\%2Fsourceforge.net\%2Fprojects\%2Fpyparsing\%2Ffiles\%2Fpyparsing\%2Fpyparsing-1.5.2\%2F\&ts=1332828745\&use\_mirror=softlayer}{Get}

\item {} 
py-radix (0.5) \href{http://py-radix.googlecode.com/files/py-radix-0.5.tar.gz}{Get}

\end{itemize}
\begin{description}
\item[{besides: it requires:}] \leavevmode\begin{itemize}
\item {} 
numpy \href{http://numpy.scipy.org/}{Get}

\item {} 
matplotlib \href{http://matplotlib.sourceforge.net/}{Get}

\item {} 
pygame \href{http://www.pygame.org/news.html}{Get}

\item {} 
profilehooks \href{http://mg.pov.lt/profilehooks/}{Get}

\end{itemize}

\end{description}

if you are in debain bases system. you can simple use

\begin{Verbatim}[commandchars=\\\{\}]
sudo apt-get install python-dev
sudo apt-get install python-numpy
sudo apt-get install python-matplotlib
sud0 apt-get install python-pygame
\end{Verbatim}


\chapter{Indices and tables}
\label{index:indices-and-tables}\begin{itemize}
\item {} 
\emph{genindex}

\item {} 
\emph{modindex}

\item {} 
\emph{search}

\item {} 
\href{http://docs.python.org/glossary.html\#glossary}{\emph{Glossary}}

\end{itemize}



\renewcommand{\indexname}{Index}
\printindex
\end{document}
